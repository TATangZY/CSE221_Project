%%%%%%%%%%%%%%%%%%%%%%%%%%%%%%%%%%%%%%%%%%%%%%%%%%%%%%%%%%%%%%%%%%%%%%%%%%%%%%%%
% Template for USENIX papers.
%
% History:
%
% - TEMPLATE for Usenix papers, specifically to meet requirements of
%   USENIX '05. originally a template for producing IEEE-format
%   articles using LaTeX. written by Matthew Ward, CS Department,
%   Worcester Polytechnic Institute. adapted by David Beazley for his
%   excellent SWIG paper in Proceedings, Tcl 96. turned into a
%   smartass generic template by De Clarke, with thanks to both the
%   above pioneers. Use at your own risk. Complaints to /dev/null.
%   Make it two column with no page numbering, default is 10 point.
%
% - Munged by Fred Douglis <douglis@research.att.com> 10/97 to
%   separate the .sty file from the LaTeX source template, so that
%   people can more easily include the .sty file into an existing
%   document. Also changed to more closely follow the style guidelines
%   as represented by the Word sample file.
%
% - Note that since 2010, USENIX does not require endnotes. If you
%   want foot of page notes, don't include the endnotes package in the
%   usepackage command, below.
% - This version uses the latex2e styles, not the very ancient 2.09
%   stuff.
%
% - Updated July 2018: Text block size changed from 6.5" to 7"
%
% - Updated Dec 2018 for ATC'19:
%
%   * Revised text to pass HotCRP's auto-formatting check, with
%     hotcrp.settings.submission_form.body_font_size=10pt, and
%     hotcrp.settings.submission_form.line_height=12pt
%
%   * Switched from \endnote-s to \footnote-s to match Usenix's policy.
%
%   * \section* => \begin{abstract} ... \end{abstract}
%
%   * Make template self-contained in terms of bibtex entires, to allow
%     this file to be compiled. (And changing refs style to 'plain'.)
%
%   * Make template self-contained in terms of figures, to
%     allow this file to be compiled. 
%
%   * Added packages for hyperref, embedding fonts, and improving
%     appearance.
%   
%   * Removed outdated text.
%
%%%%%%%%%%%%%%%%%%%%%%%%%%%%%%%%%%%%%%%%%%%%%%%%%%%%%%%%%%%%%%%%%%%%%%%%%%%%%%%%

\documentclass[letterpaper,twocolumn,10pt]{article}
\usepackage{usenix2019_v3}

% color the table
\PassOptionsToPackage{table}{xcolor}
\usepackage{colortbl}

% table number
\usepackage{caption}

% to be able to draw some self-contained figs
\usepackage{tikz}
\usepackage{amsmath}

% inlined bib file
\usepackage{filecontents}

%-------------------------------------------------------------------------------
\begin{document}
%-------------------------------------------------------------------------------

%don't want date printed
\date{}

% make title bold and 14 pt font (Latex default is non-bold, 16 pt)
\title{\Large \bf The Performance Report of a System}

%for single author (just remove % characters)
\author{
{\rm Yucheng Wang}
\and
{\rm Ziyu Tang}
} % end author

\maketitle

%-------------------------------------------------------------------------------
\begin{abstract}
%-------------------------------------------------------------------------------

\end{abstract}


%-------------------------------------------------------------------------------
\section{Introduction}
%-------------------------------------------------------------------------------


%-------------------------------------------------------------------------------
\section{Machine Description}
%-------------------------------------------------------------------------------
Characteristics of our system are:

% Please add the following required packages to your document preamble:
% \usepackage[table,xcdraw]{xcolor}
% If you use beamer only pass "xcolor=table" option, i.e. \documentclass[xcolor=table]{beamer}
\begin{table}[h]
\centering
\caption{System Configuration}
\begin{tabular}{ll}
\hline
\multicolumn{2}{c}{\cellcolor[HTML]{656565}CPU}                                                                                                  \\ \hline
\cellcolor[HTML]{9B9B9B}Mode               & AMD Ryzen 9 5900X                                                                                   \\
\cellcolor[HTML]{9B9B9B}Cycle Time         & \begin{tabular}[c]{@{}l@{}}a single FMA takes 4 cycles \\ with a throughput of 2/clock\end{tabular} \\
\cellcolor[HTML]{9B9B9B}L1d cache          & 384 KiB                                                                                             \\
\cellcolor[HTML]{9B9B9B}L1i cache          & 384 KiB                                                                                             \\
\cellcolor[HTML]{9B9B9B}L2 cache           & 6 MiB                                                                                               \\
\cellcolor[HTML]{9B9B9B}L3 cache           & 64 MiB                                                                                              \\
\cellcolor[HTML]{9B9B9B}I/O Bus            & PCIe 4.0                                                                                            \\
\cellcolor[HTML]{9B9B9B}Memory Bus         & 64 bits, 3600 mhz                                                                                   \\
\multicolumn{2}{c}{\cellcolor[HTML]{656565}Disk}                                                                                                 \\
\cellcolor[HTML]{9B9B9B}Mode               & \begin{tabular}[c]{@{}l@{}}Disk1 Samsung SSD 970 EVO Plus\\ Disk2 Samsung SSD 860\end{tabular}      \\
\cellcolor[HTML]{9B9B9B}Capacity           & 4 TB (2 * 2 TB)                                                                                     \\ \hline
\multicolumn{2}{c}{\cellcolor[HTML]{656565}RAM}                                                                                                  \\
\cellcolor[HTML]{9B9B9B}Size               & 32 GB (2 * 16 GB)                                                                                   \\
\cellcolor[HTML]{9B9B9B}Total Width        & 64 bits                                                                                             \\
\cellcolor[HTML]{9B9B9B}Type               & DDR4                                                                                                \\
\cellcolor[HTML]{9B9B9B}Memory Technology  & DRAM                                                                                                \\
\cellcolor[HTML]{9B9B9B}Speed              & 2133 MT/s                                                                                           \\
\cellcolor[HTML]{9B9B9B}Form Factor        & DIMM                                                                                                \\
\multicolumn{2}{c}{\cellcolor[HTML]{656565}Network Card}                                                                                         \\
\cellcolor[HTML]{9B9B9B}Speed              & \begin{tabular}[c]{@{}l@{}}WI-FI6, a single\\ stream at 3.5 Gbps\end{tabular}                       \\
\multicolumn{2}{c}{\cellcolor[HTML]{656565}Operating System}                                                                                     \\
\cellcolor[HTML]{9B9B9B}Release \& Version & Ubuntu 20.04                                                                                        \\
                                           &                                
\end{tabular}
\end{table}
%-------------------------------------------------------------------------------
\section{Experiments}


\subsection{CPU, Scheduling and OS Services}


\subsection{Memory}

\subsection{Network}

\subsection{File System}

%%%%%%%%%%%%%%%%%%%%%%%%%%%%%%%%%%%%%%%%%%%%%%%%%%%%%%%%%%%%%%%%%%%%%%%%%%%%%%%%
\end{document}
%%%%%%%%%%%%%%%%%%%%%%%%%%%%%%%%%%%%%%%%%%%%%%%%%%%%%%%%%%%%%%%%%%%%%%%%%%%%%%%%

%%  LocalWords:  endnotes includegraphics fread ptr nobj noindent
%%  LocalWords:  pdflatex acks
